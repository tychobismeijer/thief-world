\section{Introduction}
\label{sec:intro}
\addcontentsline{toc}{chapter}{Introduction}

The domain of cellular automata has quickly expanded over the last few decades, from a more general mathematical framework \cite{Neumann:1966:TSA:1102024} to very specific applications within traffic optimization \cite{ISI:A1995QH06900031,ISI:A1992KF03100005}, artificial life \cite{ISI:A1992JV77700063} and physics.

Our goal within this paper is to show the applicability of the cellular automata model in transporting objects within a two dimensional environment. In this sense we have designed a cellular automata that can perform this task. The physical realism of the model was not be considered for the current model, since our goal is to show that a self-organizing system can achieve such behavior. In a nutshell we would like to describe how a cellular automata model can be used to transport objects over gaps within a world in which gravity affects the movements of objects. Our study considers variations of variables such as gap size, object weight and number of cellular automata bots initially placed within the system, which have a direct effect on how efficient the object is transported to the final location. 

An important factor for an actual implementation of such a system of modular robots would be power consumption and time consumed for activating a robot to perform a moving or pushing action. Therefore we have considered this as a key point in our simulations, by taking into account the number of reconfigurations performed within the system rather than the time in which the system achieves its goal of transportation. 
