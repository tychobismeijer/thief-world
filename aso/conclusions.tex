\section{Conclusions}
\label{sec:concl}
\addcontentsline{toc}{chapter}{Conclusions}
The previous sections have shown how a cellular automata model could be used in the task of transporting a certain object within a gravity-affected world. Also, we have seen how variation in gap size, object weight and number of bots available will eventually influence the outcome of the system. In some cases we studied the interaction effects between these parameters, whereas in other cases we were interested in the overall number of reconfigurations as a definitive measure for the efficiency of the automata rules.

The current study focuses primarily on solving the somewhat simple task of transporting a single box over a single gap of height 1. However, perhaps the most important feature of our model is that its rules could be used in far more complex situations within various world configurations. For example, the current model can be used to transport a convoy of boxes aligned one after another, where multiple bots combine their pushing power to move the heavier load. Another interesting extension to the model would be to determine its use in placing certain objects at specific locations within the world so that a certain sequence of tasks is met. 
