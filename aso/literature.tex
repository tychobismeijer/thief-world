\section{Literature}
\label{sec:lit}
\addcontentsline{toc}{chapter}{Literature}

Our work is inspired by the cellular automata model of Butler et al.\cite{butler02}. The idea of using a cellular automata to control self configurable robots comes from their study, primarily based on the fact that bots move around a flat surface in their model. Another important aspect to this model is that it is also implemented in different real robots. We extend upon this by considering pushing around objects, and gravity causing bots to fall down.

As mentioned previously, the literature related to cellular automata has been rapidly growing in the past decades. However, the limited interest in practical applications had given the domain a rather exclusive touch until only recently. Among the more significant work done in the modular robotics is Yun's paper \cite{Yun:2011:OSA:2036628.2036638} that distinguishes two algorithms for modular robot manipulation within a given environment. Nevertheless, the authors do not use a cellular automata model for the reconfiguration of single bot nor do they take into consideration the practical task of transporting an object with their selected model.
